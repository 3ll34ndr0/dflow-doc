%%%%%%%%%%%%%%%%%%%%%%%%%%%%%%%%%%%%%%%%%
% Beamer Presentation
% LaTeX Template
% Version 1.0 (10/11/12)
%
% This template has been downloaded from:
% http://www.LaTeXTemplates.com
%
% License:
% CC BY-NC-SA 3.0 (http://creativecommons.org/licenses/by-nc-sa/3.0/)
%
%%%%%%%%%%%%%%%%%%%%%%%%%%%%%%%%%%%%%%%%%

%----------------------------------------------------------------------------------------
%	PACKAGES AND THEMES
%----------------------------------------------------------------------------------------

\documentclass{beamer}

\mode<presentation> {

% The Beamer class comes with a number of default slide themes
% which change the colors and layouts of slides. Below this is a list
% of all the themes, uncomment each in turn to see what they look like.

%\usetheme{default}
%\usetheme{AnnArbor}
%\usetheme{Antibes}
%\usetheme{Bergen}
%\usetheme{Berkeley}
%\usetheme{Berlin}
%\usetheme{Boadilla}
%\usetheme{CambridgeUS} %Color rojo, me gusta
%\usetheme{Copenhagen} %Ta bueno
\usetheme{Darmstadt} % Este es el 1er candidato
%\usetheme{Dresden} % Es lindo y parecido al de arriba
%\usetheme{Frankfurt}
%\usetheme{Goettingen}
%\usetheme{Hannover}
%\usetheme{Ilmenau}
%\usetheme{JuanLesPins}
%\usetheme{Luebeck}
%\usetheme{Madrid}
%\usetheme{Malmoe}
%\usetheme{Marburg}
%\usetheme{Montpellier}
%\usetheme{PaloAlto}
%\usetheme{Pittsburgh}
%\usetheme{Rochester}
%\usetheme{Singapore}
%\usetheme{Szeged}
%\usetheme{Warsaw}

% As well as themes, the Beamer class has a number of color themes
% for any slide theme. Uncomment each of these in turn to see how it
% changes the colors of your current slide theme.

%\usecolortheme{albatross}
%\usecolortheme{beaver}
%\usecolortheme{beetle}
%\usecolortheme{crane}
%\usecolortheme{dolphin}
%\usecolortheme{dove}
%\usecolortheme{fly}
%\usecolortheme{lily}
%\usecolortheme{orchid}
%\usecolortheme{rose}
%\usecolortheme{seagull}
%\usecolortheme{seahorse}
%\usecolortheme{whale}
%\usecolortheme{wolverine}

%\setbeamertemplate{footline} % To remove the footer line in all slides uncomment this line
%\setbeamertemplate{footline}[page number] % To replace the footer line in all slides with a simple slide count uncomment this line

%\setbeamertemplate{navigation symbols}{} % To remove the navigation symbols from the bottom of all slides uncomment this line
}

\usepackage{graphicx} % Allows including images
\usepackage{booktabs} % Allows the use of \toprule, \midrule and \bottomrule in tables
\usepackage{listings}
%\usepackage{url}
\usepackage{hyperref}
\usepackage[spanish]{babel}
\usepackage[utf8]{inputenc}
%----------------------------------------------------------------------------------------
%	TITLE PAGE
%----------------------------------------------------------------------------------------

\title[Flujo Digital]{Instalación y Configuración de un Flujo de Diseño Digital} % The short title appears at the bottom of every slide, the full title is only on the title page

\author{Leandro Marsó} % Your name
\institute[] % Your institution as it will appear on the bottom of every slide, may be shorthand to save space
{
Córdoba\\ % Your institution for the title page
\medskip
\textit{elleandro@gmail.com} % Your email address
}
\date{\today} % Date, can be changed to a custom date

\begin{document}


\lstset{
basicstyle=\ttfamily,                   % Code font, Examples: \footnotesize, \ttfamily
frame=none,                             % A frame around the code
tabsize=2,                              % Default tab size
captionpos=b,                           % Caption-position = bottom
breaklines=false,                        % Automatic line breaking?
breakatwhitespace=false,                % Automatic breaks only at whitespace?
showspaces=false,                       % Dont make spaces visible
showtabs=false,                         % Dont make tabls visible
showstringspaces=false,
commentstyle=\color{red},
keywordstyle=\color{blue},
}


\begin{frame}
\titlepage % Print the title page as the first slide
\end{frame}

\begin{frame}
\frametitle{Contenido} % Table of contents slide, comment this block out to remove it
\tableofcontents % Throughout your presentation, if you choose to use \section{} and \subsection{} commands, these will automatically be printed on this slide as an overview of your presentation
\end{frame}

%----------------------------------------------------------------------------------------
%	PRESENTATION SLIDES
%----------------------------------------------------------------------------------------
\section{¿Qué es el Software Libre?}
\begin{frame}
\frametitle{Definición de software libre}
«Software libre» es el software que respeta la libertad de los usuarios y la comunidad. A grandes rasgos, significa que los usuarios tienen la libertad de ejecutar, copiar, distribuir, estudiar, modificar y mejorar el software. Es decir, el «software libre» es una cuestión de libertad, no de precio.

\end{frame}
%------------------------------------------------
\begin{frame}
\frametitle{Las cuatro libertades del Software Libre}
Un programa es software libre si los usuarios tienen las cuatro libertades esenciales:
\begin{itemize}
\item La libertad de ejecutar el programa como se desea, con cualquier propósito (libertad 0).
\item La libertad de estudiar cómo funciona el programa, y cambiarlo para que haga lo que usted quiera (libertad 1). El acceso al código fuente es una condición necesaria para ello.
\item La libertad de redistribuir copias para ayudar a su prójimo (libertad 2).
\item La libertad de distribuir copias de sus versiones modificadas a terceros (libertad 3). Esto le permite ofrecer a toda la comunidad la oportunidad de beneficiarse de las modificaciones. El acceso al código fuente es una condición necesaria para ello.
\end{itemize}
\end{frame}

\section{Instalación} % Sections can be created in order to organize your presentation into discrete blocks, all sections and subsections are automatically printed in the table of contents as an overview of the talk

%------------------------------------------------

\begin{frame}[fragile]
\subsection{Instalacíon de paquetes necesarios } % A subsection can be created just before a set of slides with a common theme to further break down your presentation into chunks

\frametitle{Instalación del JDK de Java, gnucap y otras herramientas}

\noindent Desde una consola:
\begin{lstlisting}[language=bash]
 # Acceder a privilegios de root
 su
 apt-get install openjdk-6-jdk gnucap \
 perl python-matplotlib
 # Abandonar los privilegios de root
 exit
\end{lstlisting}
\end{frame}
%----------------------------------------------------

\begin{frame}
\Huge{\centerline{Fin}}
\end{frame}

%----------------------------------------------------------------------------------------

\end{document} 
